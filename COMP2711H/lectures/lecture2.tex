\documentclass[../main.tex]{subfile}
\begin{document}
\section{Lecture 2 - Predicate Logic (First-order logic)}
\subsection{Predicates}
\begin{definition}[Predicate] A predicate is a statement that may be true or false depending on its variables
\end{definition}
\subsection{Quantifiers}
\begin{definition}[Universal qunatification] $\forall x P(x)$
\end{definition}
\begin{definition}[Domain]
	The set of all possible values of a variable. \bful{Must be defined} when using quantification.	
\end{definition}
\begin{definition}[Existential quantification] $\exists  x P(x)$ (at least one x in domain)
	
\end{definition}
\subsection{Logical equivaleces}
\begin{fact} 
$\forall x P(x) \lor Q(x) \not\equiv \forall x P(x) \lor \forall x Q(x) $	
\end{fact}
\begin{fact} 
$\exists  x P(x) \land Q(x) \not\equiv \exists  x P(x) \lor \exists  x Q(x) $	
\end{fact}

\subsection{Negating Quantified Expressions}
\[
\neg \forall x P(x) \equiv \exists x \neg P(x)
.\] 
\[
\neg \exists x Q(x) \equiv \forall x \neg Q(x)
.\] 
\subsection{Nested quantifiers}
\end{document}

