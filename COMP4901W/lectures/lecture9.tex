\documentclass[../main.tex]{subfile}
\usepackage{tcolorbox}
\newtheorem{theorem}{Theorem}
\newtheorem{example}{Example}[theorem]
\newtheorem{task}{Task}
\newtheorem{definition}{Definition}
\begin{document}
\section{Lecture 9 - Centralized Ledger}
Let there is a central bank B to keep track of the history of transactions \textbf{done by the users} to prevent the problem of double spending. Let block be a sequence(\textbf{Merkle tree}) of transactions of size at most b, which will be \textbf{created and published by the bank}. Problem:
\begin{enumerate}
	\item How to know if $b_{i}$ was created by the bank? Soln: The bank signs every block
	\item How to ensure the bank does not change the history?
\end{enumerate}
\subsection{Viewpoint of a node}
\begin{enumerate}
	\item Case 1: Receive a transaction. Steps:
		\begin{enumerate}
			\item Verify if it is valid
			\item input $\ge$  output
			\item No double spending: input has happened before in the block chain
			\item Propagate to the neighbours
		\end{enumerate}
	\item Case 2: Receive a block $B_{i}$: Steps:
		\begin{enumerate}
			\item Verify signature of the bank
			\item Validate every transactions in $B_{i}$ 
			\item If $B_{i}$ is valid, add it to my blockchain and send to neighbours
		\end{enumerate}
\end{enumerate}
\subsection{Viewpoint of the bank}
\begin{enumerate}
	\item Maintain a copy of the blockchian
	\item Maintain a new block
	\item Listen for transactions
\end{enumerate}
\end{document}
