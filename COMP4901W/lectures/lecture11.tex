
\documentclass[../main.tex]{subfile}
\usepackage{tcolorbox}
\begin{document}
\newtheorem{theorem}{Theorem}
\newtheorem{example}{Example}[theorem]
\newtheorem{task}{Task}
\newtheorem{definition}{Definition}
\begin{document}
\section{Lecture 11 - Mining rewards and forks}
\begin{definition}[Transaction fee]
 $\sum$ input - $\sum$ output
\end{definition}
Why should a miner mine a non-empty block? Miners gain transaction fees
after mining blocks. The difference $\sum$ inputs -  $\sum$ outputs is paid to the miner. 
Thus, if you want to add your transaction earlier, pay more transaction fee to lure miners.
\subsection{Problem with forks}
The transactions that are not in consensus chains are neglected and miners get no reward.   
Hence, miners are aware of the existence of forks, and will put the transactions in the chain that is less likely to be a consensus branch. This doesn't create the double-spending problem as the memory pool will only check the same chain. Besides, after being mined, the transactions in conflict will be removed from the memory pool (Lecture 10). 
\subsection{Double spending attack}
Scenario: bit-coin based vending machine
\subsubsection{First atack: Immediate double-spending}
\begin{enumerate}
	\item create one transaction $T_{1}$, with 1 BTC as the input, 0.99 as the output, and then receive the product
	\item create another transaction $T_{2}$, with 1 BTC as the input, 0.95 BTC as the output, that gives money to herself, immediately, which creates a fork.
	\item if the miner includes $T_{2}$ in a block, which the transaction fee is 0.05BTC > 0.01 BTC, then the attack was successful, as the player can get both the money and the product.
\end{enumerate}
	Soln: Wait for $T_{1}$ to be mined
	\subsubsection{Second attack: Double-spending in a fork}
	Goal: Resolve and mine all the honest transactions. Problem: some blocks can be reverted (i.e. $T_{1}$ ) and removed. Soln: Wait for confirmation(6 blocks added after the block). Problem to soln: need to wait for an hour (not friendly to immediate transaction)
	\subsection{Adjusting difficulty}
	Let $d$ be the constant that $h(B) < d$ where B is the block. The probability of success is $d / 2^{256}$ (note: 256 bits is the max size of each block so as to uphold the latency rate)
	Also, we can ask the miners to check the timestamp such that if the timestamp is too old, then ignore it. (remark: the more far away, the longer the transaction chain, the harder it is to find a nonce)  

	Find the average mining time $\mu$. If $\mu > 10 $, then decrease d and increase difficulty. Else, increase d and decrease difficulty. 
\end{document}


