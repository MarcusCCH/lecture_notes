
\documentclass[../main.tex]{subfile}
\usepackage{tcolorbox}
\begin{document}
\newtheorem{theorem}{Theorem}
\newtheorem{example}{Example}[theorem]
\newtheorem{task}{Task}
\newtheorem{definition}{Definition}
\begin{document}
\section{Lecture 12 - Centralization and Scripts}
\subsection{51\% attack}
Imagine a malicious miner with more than half of the hash power. Then, he can revert all the honest minings that follow the protocol: adding to the consensus branch, by keep on adding to the new fork and lengthen it.
\subsection{Scripts}
Scripts are to impose certain requirements to choosing blocks
Suppose two players: A and B. They have a phone call and 1 BTC for each transaction. A and B do not trust each other. Then, we can use the following protocol, which require signatures from both A and B or signature from A only when block number > C (just to prevent the case where B doesnt sign, and A lost 100-i BTC)
\begin{enumerate}
	\item create first despoit (input: 100, output:100)
	\item for each minute i, A pays i BTC to Bob, 100 - i BTC to herself.
	\item B signs the last contract and pending to be mined.
\end{enumerate}
\end{document}
