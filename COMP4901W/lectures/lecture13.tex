\documentclass[../main.tex]{subfile}
\usepackage{tcolorbox}
\begin{document}
\newtheorem{theorem}{Theorem}
\newtheorem{example}{Example}[theorem]
\newtheorem{task}{Task}
\newtheorem{definition}{Definition}
\section{Lecture 13 - Two-way Payment Channels}
\subsection{One-way payment channel}
Properties:
\begin{enumerate}
	\item flow in one direction(A $\to$ B)
	\item B will sign and close the transaction
	\item The transaction expires at time t
	\item The channel has a fixed capacity
	\item only two transactions (base deposit, final transaction) are added to the blockchain. Thus, only two Tx fees are paid for k payments
\end{enumerate}
\subsection{Two-way payment channel}
Problem of cheating:
 B published $T_{x_{i}}$ but $\exists  T_{x_{j}} j > i$ that pays less to the player, meaning the B is not honouring the refund that owes to A.
Solution: Change the protocol such that B has to sign the index of every transaction. 
Thus, cheating now means that B's signature is on an index > current transaction's index

4 channels, 2 each one-way channel
Scenario:
\begin{enumerate}
	\item A pays 75 to B. A pay 75 in c1  (c1's capacity = 25)
	\item B pays 80 to A. B refunds 75 in c1, and pay 5 in c2 (c1's capacity = 100, c2's capacity = 95)
	\item A pays 50 to B. A refunds 5 in c2, and pay 45 in A(c1's capacity = 55, c2's capacity = 100)
\end{enumerate}
\end{document}


